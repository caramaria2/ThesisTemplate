	\documentclass[
    headings=small,
    version=first,
	oneside,        					% print only on the right pages
    listof=totoc,version=first,     		% includes list of figures and list of tables in table of contents
    bibliography=totoc,version=first,   % includes bibliography in table of contents 
    headsepline,    					% divide under the topline
    12pt
    ]{scrbook}

%\usepackage{indentfirst}
\usepackage{a4}  
\usepackage[left=3cm,right=3cm,top=3cm,bottom=4cm]{geometry}
\usepackage[english, ngerman]{babel}
\usepackage[utf8]{inputenc} %damit k{\"o}nnen Umlaute ganz normal geschrieben werden. 

\usepackage{listings} %Fuer Codelistings

\usepackage{subfigure} %f{\"u}r mehrteilige Grafiken
\usepackage{epsfig}    %damit funktioniert das einbinden von grafiken {\"u}ber epsfig.
\usepackage{graphicx}     % zum einbinden von grafiken
\graphicspath{{grafiken}{../}{kapitel}} 
\usepackage{multirow}    
\usepackage{longtable}	
\usepackage{rotating}	%enable landscape pages
\usepackage{framed}
\usepackage{scrlayer-scrpage}     % paket f{\"u}r kopf- und fu{\ss}zeilen
\pagestyle{scrheadings}   % kopzeilenseitenstil

\usepackage{float} % adds location parameter [H] to [htb] to force figures and tables to a location, no floating



%\usepackage[
%a4paper=true, % a4
%pdftitle={The title of this paper}, %title to display for pdfs
%pdfauthor={The Author}, %set author
%dvipdfmx, % must be used if pdf is created from the dvi file rather than by running pdflatex. Otherwise links won't work
%colorlinks,% set link color for all links to black, i.e. invisible but working links
%citecolor=black,%
%filecolor=black,%
%linkcolor=black,%
%urlcolor=black, %
%plainpages=false, %used with pdfpagelabels to make pdf readers show roman and arabic page numbers correctly
%pdfpagelabels, %used with plainpages=false to make pdf readers show roman and arabic page numbers correctly
%pagebackref=true %false: no back-references in bibliography to citations in text
%]{hyperref} %include hyperlinks for citations, urls, and references
%CAUTION: breaks color settings in dvi, pdf works fine
%%%%end: use hyperlinks


\usepackage{setspace}
\usepackage{url}          % fuer urls: schreibweise ist z.B. \url{http://www.uni-mannheim.de}
\usepackage{nomencl} % um Abkürzungen aufzunehmen: \abbrev{PDA}{personal digital assistant}
\let\abbrev\nomenclature
\renewcommand{\nomname}{List of Abbreviations} %Titel der Liste hier ändern, eine von zwei Optionen nutzen
%\renewcommand{\nomname}{Abkürzungsverzeichnis} %Titel der Liste hier ändern, eine von zwei Optionen nutzen
\setlength{\nomlabelwidth}{.25\hsize}
\renewcommand{\nomlabel}[1]{#1 \dotfill}
\setlength{\nomitemsep}{-\parsep}
\makenomenclature
\newcommand{\Listofabbrev}{
\printnomenclature
\newpage
}




%Inhaltsverzeichnis
\usepackage
%[
%tocfullflat,			%alle Eintraege left-aligned, alternativ: tocflat
%tocbreakscareless		%page break between toc entry allowed
%]
{tocstyle}			%mehr Kontrolle ueber Inhaltsverzeichnis
\usetocstyle{allwithdot}	%ensure there are dots in table of contents, alternativ: noonewithdot, nopagecolumn
% docu http://www.tug.org/texlive/devsrc/Master/texmf-dist/source/latex/koma-script/tocstyle.dtx
\setcounter{secnumdepth}{3} %numbering up to fifth level in table of contents
\setcounter{tocdepth}{3} %show entries in toc up to fifth level (subsubsubsubsection)


\setlength{\parindent}{0pt}		% Einzug fuer neuen Absatz
\setlength{\parskip}\medskipamount % Abstand neuer Absatz, besser als explizite Angabe in pt

\onehalfspacing %Zeilenabstand 1,5


% kapitel{\"u}berschriften in schriftart mit serifen
\setkomafont{sectioning}{\normalfont\normalcolor\bfseries}

% gestaltung der kopfzeilen
\ohead{\pagemark}
\ifoot{}
\cfoot{}
\ofoot{} 
\cohead{}
\ihead{\headmark}
\setkomafont{pagehead}{\normalfont\bfseries}
\setkomafont{pagenumber}{\normalfont\bfseries}
\automark{chapter}
\automark*{section}

% ----- ende der pr{\"a}ambel ----------------------------------






\begin{document}  % dokument f{\"a}ngt an
\selectlanguage{english} %englische Silbentrennung, fuer deutsche Arbeiten: \selectlanguage{ngerman}
\frontmatter      % vorspann, kapitel r{\"o}misch nummeriert

\newgeometry{margin=3cm}
% Die Titelseite der Arbeit

\begin{titlepage}

\begin{center} % zentrieren

  % Logo
  \begin{figure}[ht]
    \centering
    \includegraphics[width=.5\textwidth]{grafiken/unilogo.png}
  \end{figure}
  
  % Vertikaler Zwischenraum
  \bigskip
  \vfill 
  \begin{framed}
    \begin{center}
     \textsc{{\LARGE Thesis title}}  \\
      \bigskip
	%Here goes the subtitle
      SUBTITLE\\
      \bigskip
      \textbf{Master Thesis}
    \end{center}
   \end{framed}
    \vfill
    \vfill
  
%Students Data
  \begin{tabular*}{0.62\textwidth}{r@{\extracolsep{\fill}}l}
   Submitted: &\hspace{1cm} December 2020\\\\
   By: &\hspace{1cm} Cara Maria Damm\\
		&\hspace{1cm}  cadamm@mail.uni-mannheim.de\\
    &\hspace{1cm} born June 18 1994\\
    & \hspace{1cm} in Karlsruhe, Germany\\
    \\
    Matriculation Number: &\hspace{1cm} 1631263\\
    \\
     Supervisor: & \hspace{1cm} Philipp Hoffmann\\
     Reviewer: &\hspace{1cm} Prof. Dr. Armin H. Heinzl\\

  \end{tabular*}
    
  \vfill
  \vfill

  
  \rule{\textwidth}{.4pt}\\ % vertikale Linie
  University of Mannheim\\
  Chair of General Management and Information Systems\\
 68131 Mannheim\\
  Phone: +49 (0) 621 181 1691, Fax: +49 (0) 621 181 1692\\
  Homepage: https://www.bwl.uni-mannheim.de/heinzl/
\end{center}

\end{titlepage} 
%//End of Title

 
     % titelseite einbinden
\restoregeometry



\chapter{Abstract}
\thispagestyle{empty}

The abstract offers a brief description of your thesis and a concise summary of its conclusions. Be sure to describe the subject and focus of your work. Please avoid symbols, foreign words, formulas, diagrams and other illustrative materials, lengthy explanations, or opinions. Do not exceed 200 words.

\tableofcontents            % inhaltsverzeichnis

\listoffigures              % abbildungsverzeichnis
\listoftables               % tabellenverzeichnis


\addcontentsline{toc}{chapter}{\nomname} %abkuerzungen ins inhaltsverzeichnis

\Listofabbrev % liste der abkuerzungen erstellen



\mainmatter       % hauptteil, kapitel lateinisch nummeriert
%\include{kapitel/content}
\chapter{Introduction}

This exemplary document serves as a guideline for both the outline and format of your work. In this section, you introduce the topic. The research problem should be derived from real-world situations and research to show awareness of the issue. Therefore, you should present the business problem, the scientific problem, the paper’s objectives, and the expected contributions. A good guideline on the content and structure of a research paper is provided by Venkatesh (2011, pp. 46–54) \cite{tannenbaum}. In the following chapter, the citation style for your thesis will be explained and presented. Here comes a change

\section{Objectives of the Thesis}
\hspace{1cm}
In order to create a common understanding of the research project described in your thesis, it is important to highlight the objectives of this thesis. The formulation of research questions is helpful to conclude the objectives in a concrete manner.
\section{Structure of the Thesis}
The explanation of the structure of your thesis helps a reader to follow your concept and offers a rough overview of the chapters of your thesis.
%\selectlanguage{ngerman} % jetzt sprechen wir deutsch.
\chapter{Implementation}

We generally recommend the following directory structure for a software project:

\begin{itemize}
\item projektname
  \begin{itemize}
  \item bin
  \item doc
  \item lib
  \item src
  \end{itemize}
\end{itemize}

The software should be saved in the package de.uni-mannheim.informatik.swt.projektname under src.

The source code should be entirely in English, including comments, names of functions, variables, menu items in the user interface, short explanations and program output.

The code should be formatted and emitted in the \texttt{Courier New font}, as in the following example:

\lstinputlisting[
  language=Java, numbers=left, stepnumber=5, firstnumber=1, breaklines=true, 
  basicstyle=\footnotesize,
  numberstyle=\tiny,
  caption={GuestbookForm.java},
  captionpos=b,
  label=GuestbookForm
]
{code/GuestbookForm.java.txt}

%\selectlanguage{english} % jetzt sprechen wir wieder englisch
%\selectlanguage{ngerman} % wenn im Inhaltsverzeichnis Appendix stehen soll, muss Engl gewaehlt sein, fuer Anhang Deutsch
%\appendix
\backmatter
\pagenumbering{roman}
\setcounter{page}{7}

%Bibliographie
%waehle einen der folgenden 4 Eintraege
%\bibliographystyle{literatur/natdin} %DIN Style Literaturverzeichnis, comment out pagebackref
\bibliographystyle{literatur/IEEEtran} % IEEE Style Literaturverzeichnis
\bibliographystyle{literatur/apa} % IEEE Style Literaturverzeichnis
%\bibliographystyle{literatur/natdinCustomized} %DIN Style Literaturverzeichnis + Punkt hinter jeder Literaturangabe -> low level config fuer Zitate: natdin.cfg im Projektordner, comment out pagebackref
%\bibliographystyle{literatur/natdinCustomizedEnglish} %DIN Style auf Englisch getrimmt mit Punkt hinter Literaturangabe -> low lovel config fuer Zitate: natdin.cfg im Projektordner, pagebackref kann damit nicht genutzt werden
%\bibliographystyle{aer} % alternativ auch apalike, aer, apalike2... s.a. http://web.reed.edu/cis/Help/LaTeX/bibtexstyles.html
%\bibliographystyle{plain} % very nice bib style
%note on aer: does not like inbook entries
%\bibliographystyle{natdin}
%\bibliographystyle{apa}

\bibliography{literatur/lit} %Pfad zur bib-Datei

%appendices can be defined here, the appendix structure has to be added manually to the toc (table of contents)
%\clearpage  %toc new page
%\addcontentsline{toc}{chapter}{Appendix} %add chapter to toc
\appendix
\addpart{\appendixname}
\chapter*{First class of appendices}
\section{Some appendix}

Appendices contain any further information, which is noteworthy but not necessarily needed to describe in the main part of your thesis. Often complex tables and figures created during your research project will be presented in an Appendix


% \input{kapitel/methodologyQuestionMappingTable}
% 
% \input{kapitel/question_origins}
% \include{extern/fragebogenE}
% \include{extern/fragebogenD}
% 
% \input{kapitel/invitationLetters}
% 
% 
% \include{kapitel/coverageTable}

%Eidesstattliche Erklaerung
\chapter*{\large Declaration of Authorship / Eidesstattliche Erkl\"{a}rung}
\pagestyle{empty}
I hereby declare that the work presented in this thesis is my own and that I have not called upon the help of a third party. In addition, I affirm that neither I nor anybody else has previously submitted this work or parts of it to obtain credits elsewhere. I have clearly marked and acknowledged all quotations and references that have been taken from the works of others. All secondary literature and other sources are marked and listed in the bibliography. The same applies to all charts, diagrams and illustrations as well as to all Internet resources. Moreover, I consent to my paper being electronically stored and sent anonymously in order to be checked for plagiarism. I know that if this declaration is not made, the paper may not be graded.

\vspace{-0.3cm}

\hrulefill

\vspace{-0.2cm}
Hiermit versichere ich, dass diese Abschlussarbeit von mir persönlich verfasst
ist und dass ich keinerlei fremde Hilfe in Anspruch genommen habe. Ebenso
versichere ich, dass diese Arbeit oder Teile daraus weder von mir selbst noch
von anderen als Leistungsnachweise andernorts eingereicht wurden. Wörtliche oder
sinn\-gemäße Übernahmen aus anderen Schriften und Veröffentlichungen in gedruckter
oder elektronischer Form sind gekennzeichnet. Sämtliche Sekundärliteratur und
sonstige Quellen sind nachgewiesen und in der Bibliographie aufgeführt. Das
Glei\-che gilt für graphische Darstellungen und Bilder sowie für alle
Internet-Quellen.

Ich bin ferner damit einverstanden, dass meine Arbeit zum Zwecke eines
Plagiatsabgleichs in elektronischer Form anonymisiert versendet und gespeichert
werden kann. Mir ist bekannt, dass von der Korrektur der Arbeit abgesehen werden
kann, wenn die Erklärung nicht erteilt wird.

\vspace{2.0cm}

Mannheim, \today \hspace{7cm} Signature\\

%Abtretungs Erklaerung
\chapter*{\large Assignment of Usage Rights / Abtretungserkl\"arung}
\pagestyle{empty}
I hereby grant the University of Mannheim, Chair of Software Engineering, Prof. Dr. Colin Atkinson, extensive, exclusive, unlimited and unrestricted usage rights to the work described in this thesis. 
This includes the right to use the results in research and teaching. In particular, this includes the right to reproduce, distribute, translate, change and transfer the results to third parties, as well as any further results derived from them.
Whenever the results of my work are used in their original form or in a revised version, I will be mentioned by name as a co-author, in compliance with the copyright rules. This assignment does not include any commercial use.

\vspace{-0.3cm}

\hrulefill

\vspace{-0.2cm} 
Hinsichtlich meiner Masterarbeit räume ich der Universität Mannheim/Lehrstuhl für Softwaretechnik,
Prof. Dr. Colin Atkinson, umfassende, ausschließliche unbefristete und
unbeschränkte Nutzungsrechte an den entstandenen Arbeitsergebnissen ein.

Die Abtretung umfasst das Recht auf Nutzung der Arbeitsergebnisse in Forschung
und Lehre, das Recht der Vervielfältigung, Verbreitung und Übersetzung sowie
das Recht zur Bearbeitung und Änderung inklusive Nutzung der dabei
entstehenden Ergebnisse, sowie das Recht zur Weiterübertragung auf Dritte.

Solange von mir erstellte Ergebnisse in der ursprünglichen oder in
überarbeiteter Form verwendet werden, werde ich nach Maßgabe des Urheberrechts
als Co-Autor namentlich genannt. Eine gewerbliche Nutzung ist von dieser
Abtretung nicht mit umfasst.
\bigskip

\vspace{2.0cm}

Mannheim, \today \hspace{7cm} Signature\\


\end{document}
