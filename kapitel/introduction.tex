\chapter{Introduction}

This exemplary document serves as a guideline for both the outline and format of your work. In this section, you introduce the topic. The research problem should be derived from real-world situations and research to show awareness of the issue. Therefore, you should present the business problem, the scientific problem, the paper’s objectives, and the expected contributions. A good guideline on the content and structure of a research paper is provided by Venkatesh (2011, pp. 46–54) \cite{beaulieu_road_2012}. In the following chapter, the citation style for your thesis will be explained and presented.

\section{Objectives of the Thesis}
\hspace{1cm}
In order to create a common understanding of the research project described in your thesis, it is important to highlight the objectives of this thesis. The formulation of research questions is helpful to conclude the objectives in a concrete manner.
\section{Structure of the Thesis}
The explanation of the structure of your thesis helps a reader to follow your concept and offers a rough overview of the chapters of your thesis.